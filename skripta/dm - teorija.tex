 \documentclass{article}
\usepackage[total={7in, 10in}]{geometry}
\usepackage[dvipsnames]{xcolor}
\usepackage{amsmath}
\usepackage[unicode]{hyperref}
\usepackage{fancyhdr}
\usepackage{amssymb}
\usepackage{amsthm}
\usepackage[croatian]{babel}
\usepackage{multirow}


\definecolor{carminered}{rgb}{1.0, 0.0, 0.22}
\definecolor{capri}{rgb}{0.0, 0.75, 1.0}
\definecolor{brightlavender}{rgb}{0.85, 0.7, 0.95}

\title{\textbf{Diskretna matematika 120 - teorija}}
\author{Q}
\date{\today}
\begin{document}


\pagecolor{black}
\color{white}
\maketitle
\color{brightlavender}
\section{Skupovi i funkcije}
\color{white}
\pagenumbering{roman}
\fancyfoot[C]{Page \thepage\ of \pageref{LastPage}}

\begin{enumerate}

\item Objasniti što podrazumijevamo pod pojmom skupa te na koje načine možemo skup zadati.

Pod pojmom skup podrazumijevamo bilo koju množinu elemenata.

\item Navesti definicije pojmova: podskup, jednakost skupova, pravi podskup, prazan skup, partitivni skup, disjunktni skupovi.

Podskup ... skup $A$ je podskup skupa $B$ ako za bilo koji $x \in A$ također vrijedi $x \in B$ i to pišemo kao $A \subseteq B$.

Jednakost ... kažemo da su skupovi $A$ i $B$ jednaki ako vrijedi $A\subseteq B$ i $B\subseteq A$ te pišemo $A=B$.

Pravi podskup ... kažemo da je skup $A$ pravi podskup skupa $B$ ako vrijedi $A\subseteq B$ i $A\neq B$ te pišemo $A\subset B$.

Prazan skup ... onaj koji ne sadrži elemente, pišemo $\emptyset$. Za svaki skup $A$ vrijedi $\emptyset\subseteq A$.

Partitivni skup ... za bilo koji skup $A$ možemo definirati skup koji kao svoje elemente sadrži podskupove od $A$, skup svih podskupova od $A$ zovemo partitivni skup i pišemo $\mathcal{P}(A)$.

Disjunktni skupovi ... za dva skupa $A$ i $B$ vrijedi da su disjunktni ako je njihov presjek prazan skup.

\item Definirati skupovne operacije: unija, presjek, komplement, razlika, simetrična razlika, te navesti svojstva svih tih operacija.

Unija ... za dva skupa $A$ i $B$ sadržanih u univerzalnom skupu definiramo uniju skupova $A$ i $B$ kao skup svih elemenata $x$ za koje vrijedi $x\in A$ ili $x\in B$, takav skup označavamo sa $A\cup B$.

Presjek ... za dva skupa $A$ i $B$ sadržanih u univerzalnom skupu definiramo presjek skupova $A$ i $B$ kao skup svih elemenata $x$ za koje vrijedi $x\in A$ i $x\in B$, takav skup označavamo sa $A\cap B$.

Komplement ... za skup $A$ koji je sadržan u univerzalnom skupu definiramo komplement skupa $A$ kao skup svih elemenata $x$ za koje vrijedi $x\notin A$, a označavamo ${A}^\complement$ ili $\overline{A}$.

Razlika ... za dva skupa $A$ i $B$ sadržanih u univerzalnom skupu definiramo razliku skupova $A$ i $B$ kao skup svih elemenata $x$ za koje vrijedi $x\in A$ i $x\notin B$, te pišemo $A\setminus B$.

Simetrična razlika ... za dva skupa $A$ i $B$ sadržanih u univerzalnom skupu definiramo simetričnu razliku skupova $A$ i $B$ kao skup svih elemenata $x$ za koje vrijedi $x\in A$, $x\in B$ i $x\notin A\cap B$, te pišemo $A\triangle B$.

Neka su $A$, $B$ i $C\in 2^{X}$, tada vrijede sljedeća svojstva operacija nad skupovima:
\begin{itemize}
\item idempotentnost
$$A\cup A = A , \quad A\cap A = A$$
\item asocijativnost
$$A\cup(B\cup C) = (A\cup B)\cup C$$
$$A\cap(B\cap C) = (A\cap B)\cap C$$
\item komutativnost
$$A\cup B = B\cup A, \quad A\cap B = B\cap A$$
\item distributivnost
$$A\cap(B\cup C) = (A\cap B)\cup(A\cap C)$$
$$A\cup(B\cap C) = (A\cup B)\cap(A\cup C)$$
\item De Morgan
$$\overline{A\cup B} = \overline{A}\cap\overline{B}$$
$$\overline{A\cap B} = \overline{A}\cup\overline{B}$$
\item ostala pravila
$$A\cup\emptyset = A, \quad A\cap X = A$$
$$A\cup\overline{A} = X, \quad A\cap\overline{A} = \emptyset$$
$$\overline{\overline{A}} = A$$
\end{itemize}

\item Dokazati De Morganove formule za dva proizvoljna skupa.

Prema De Morganovim formulama vrijedi da za dva skupa $A$, $B \in 2^{X}$ vrijedi:
$$\overline{A\cup B} = \overline{A}\cap\overline{B}$$
$$\overline{A\cap B} = \overline{A}\cup\overline{B}$$
Dokažimo prvo prvu formulu.

Neka je $x\in\overline{A\cup B}$, odnosno $x\notin A\cup B$. Tada također vrijedi $x\notin A$ i $x\notin B$, drugim riječima $x\in \overline{A}$ i $x\in\overline{B}$ ali to samo znači $x\in\overline{A}\cap\overline{B}$ pa time vrijedi $\overline{A\cup B}\subseteq\overline{A}\cap\overline{B}$.

Sada neka je $x\in\overline{A}\cap\overline{B}$, odnosno $x\in\overline{A}$ i $x\in\overline{B}$. Uočavamo da onda vrijedi i $x\notin A$ i $x\notin B$, odnosno $x\notin A\cup B$, drugim riječima $x\in\overline{A\cup B}$ pa time vrijedi $\overline{A}\cap\overline{B}\subseteq\overline{A\cup B}$.

Pokazali smo da vrijedi
$$\overline{A\cup B}\subseteq\overline{A}\cap\overline{B}\quad i \quad \overline{A}\cap\overline{B}\subseteq\overline{A\cup B}$$
Te zaključujemo
$$\overline{A\cup B} = \overline{A}\cap\overline{B}$$
Dokažimo sada drugu formulu.

Neka je $x\in\overline{A\cap B} \Longrightarrow x\notin A\cap B \Longrightarrow x \notin A$ ili $x\notin B\Longrightarrow x\in\overline{A}$ ili $x\in\overline{B}\Longrightarrow x\in\overline{A}\cup\overline{B}$ pa vrijedi $\overline{A\cap B}\subseteq\overline{A}\cup\overline{B}$.

Sada neka je $x\in\overline{A}\cup\overline{B}\Longrightarrow x\in\overline{A}$ ili $x\in\overline{B}\Longrightarrow x\notin A$ ili $x\notin B\Longrightarrow x\notin A\cap B\Longrightarrow x\in\overline{A\cap B}$ pa vrijedi $\overline{A}\cup\overline{B}\subseteq\overline{A\cap B}$.

Pokazali smo da vrijedi
$$\overline{A\cap B}\subseteq\overline{A}\cup\overline{B} \quad i \quad \overline{A}\cup\overline{B}\subseteq\overline{A\cap B}$$
Te zaključujemo
$$\overline{A\cap B} = \overline{A}\cup\overline{B}$$

\item Definirati pojam Kartezijevog produkta  $n$ - proizvoljnih skupova.

Ako su $A_1, A_2, \ldots, A_n$ neprazni skupovi, onda definiramo Kartezijev produkt
$$A_1 \times A_2 \times\ldots\times A_n$$
kao skup svih uređenih parova $(a_1, a_2, \ldots, a_n)$ takvih da je $a_k\in A_k$ za sve $k = 1, 2, \ldots, n$

\item Definirati pojmove konačan skup, beskonačan skup, ekvipotentni skupovi, prebrojiv skup, neprebrojiv skup.

Konačan skup ... za neprazan skup $A$ kažemo da je konačan ako postoji $n\in\mathbb{N}$ i bijekcija takva da $$f: \{1, 2, \ldots, n\}\to A$$

Beskonačan skup ... za skup $A$ kažemo da je beskonačan ako nije konačan.

Ekvipotentni skupovi ... Skup $A$ je ekvipotentan sa skupom $B$ ako postoji bijekcija takva da $f: A\to B$.

Prebrojivi skup ... za beskonačan skup $A$ kažemo da je prebrojiv ako se skup njegovih elemenata može poredati u beskonačan niz $ A = \{a_1, a_2, \ldots\}$.

Neprebrojiv skup ... za beskonačan skup $A$ kažemo da je neprebrojiv ako se ne može poredati u niz.

\item Dokazati da je funkcija $f: A \to B$ injektivna akko je surjektivna, pri čemu su $A$ i $B$ konačni skupovi s jednakim brojem elemenata.

Dokažimo prvo u jednom smjeru.

Neka je $f$ injektivna funkcija, tada skupovi $A$ i $f(A)$ imaju isti broj elemenata. Također znamo da vrijedi $|A| = |B|$ pa znamo $|f(A)| = |B|$.

Za svaku funkciju vrijedi da je slika funkcije podskup kodomene odnosno $f(A)\subseteq B$, ali uz dodatan uvjet $|f(A)| = |B|$ sada imamo $f(A) = B$, odnosno funkcija je surjekcija.

Sada dokažimo u drugom smjeru.

Neka je $f$ surjektivna funkcija, tada znamo da je $f(A) = B$, odnosno $|f(A)| = |B|$. Uz uvjet $|A| = |B|$ sada imamo $|f(A)|=|A|$, drugim riječima domena i slika funkcije imaju jednak broj elemenata pa je funkcija injekcija.

\item Definirati pojam kardinalnog broja skupa te prokomentirati ekvipotentnost kod konačnih i beskonačnih skupova.

Za konačan skup $A$ s $n$ elemenata definiramo kardinalni broj skupa kao $|A| = n$.

\end{enumerate}

\color{brightlavender}
\section{Matematička logika}
\color{white}

\begin{enumerate}
\item Definirati pojmove: sud, semantička vrijednost suda.

Sud ... bilo koja rečenica koja je ili istinita ili lažna. Svakom sudu $A$ pridružujemo vrijednosti $\top$ ili $\bot$ (redom istina i lažna).

Semantička vrijednost ... vrijednost istinitosti suda, semantičku vrijednost suda $A$ označavamo sa $\tau (A)$ čija vrijednost može biti $\tau (A) = \top$ ili $\tau (A) = \bot$.

\item Definirati operacije sa sudovima: negacija, konjunkcija, disjunkcija, ekskluzivna disjunkcija, implikacija, ekvivalencija, Shefferova operacija, Lukasiewiczeva operacija.

Negacija ... negacija suda $A$ je sud koji označavamo sa $\neg A$ (čitamo: ne A), a pripadajuća tablica istinitosti je
\begin{displaymath}
\begin{array}{|c|c|}
A & \neg A\\
\hline
\top & \bot\\
\bot & \top\\
\end{array}
\end{displaymath}

Konjunkcija ... konjunkcija sudova $A$ i $B$ je sud koji označavamo sa $A\land B$ (čitamo: A i B).

Disjunkcija ... disjunkcija sudova $A$ i $B$ je sud koji označavamo sa $A\lor B$ (čitamo: A ili B).

Ekskluzivna disjunkcija ... ekskluzivna dijsunkcija sudova $A$ i $B$ je sud koji označavamo sa $A\veebar B$ (čitamo: ili A ili B).

Implikacija ... implikacija sudova $A$ i $B$ je sud koji označavamo sa $A\Rightarrow B$ (čitamo: iz A slijedi B).

Ekvivalencija ... ekvivalencija sudova $A$ i $B$ je sud koji označavamo sa $A\Leftrightarrow B$ (čitamo: A je ekvivalentan sa B).

Shefferova operacija ... shefferova operacija između sudova $A$ i $B$ je sud koji označavamo sa $A\uparrow B$ (čitamo: A šefer B), te ima značenje nije istodobno $A$ i $B$.

Lukasiewiczeva operacija ... lukasiewiczeva operacija između sudova $A$ i $B$ je sud koji označavamo sa $A\downarrow B$ (čitamo: A lukasijevič B), te ima značenje niti je $A$ niti je $B$.

Pripadajuća tablica istinitosti za operacije konjunkcije, disjunkcije, ekskluzivne disjunkcije, implikacije, ekvivalencije, Shefferove operacije i Lukasiewiczeve operacije
\begin{displaymath}
\begin{array}{|c|c|c|c|c|c|c|c|c|}
A & B & A\land B & A\lor B & A\veebar B & A\Rightarrow B & A\Leftrightarrow B & A\uparrow B & A\downarrow B\\
\hline
\bot & \bot & \bot & \bot & \bot & \top & \top & \top & \top\\
\bot & \top & \bot & \top & \top & \top & \bot & \top & \bot\\
\top & \bot & \bot & \top & \top & \bot & \bot & \top & \bot\\
\top & \top & \top & \top & \bot & \top & \top & \bot & \bot\\
\end{array}
\end{displaymath}


\item Definirati pojam logičke ekvivalentnosti dviju logičkih formula te navesti poredak logičkih operacija po snazi vezivanja.

Logička ekvivalentnosti ... Kažemo da su dvije formule $P$ i $Q$ algebre sudova logički ekvivalentne ako imaju isti broj varijabla i podjednake tablice istinitosti, te pišemo $P\equiv Q$.

Poredak logičkih operacija po opadajućoj snazi vezivanja je
$$\neg,\quad\land,\quad\lor,\quad\Rightarrow,\quad\Leftrightarrow$$

\item Definirati pojmove generatora algebre sudova i baze algebre sudova.

Generator algebre sudova ... sistem izvodnica (generatora) algebre sudova je skup Booleovih operacija algebre sudova pomoću kojih se može napisati bilo koja formula algebre sudova.

Baza algebre sudova ... Minimalni sistem izvodnica algebre sudova zovemo bazom algebre sudova.

\item Iskazati i dokazati teorem koji daje svojstva operacija sa sudovima.

Vrijede sljedeća pravila algebre sudova
\begin{itemize}
\item idempotentnost operacija disjunkcije i konjunkcije
$$A\lor A\equiv A,\quad A\land A\equiv A$$
\item asocijativnost
$$(A\lor B)\lor C\equiv A\lor (B\lor C)$$
$$(A\land B)\land C\equiv A\land (B\land C)$$
\item komutativnost
$$A\lor B\equiv B\lor A,\quad A\land B\equiv B\land A$$
\item distributivnost
$$A\land (B\lor C)\equiv (A\land B)\lor (A\land C)$$
$$A\lor (B\land C)\equiv (A\lor B)\land (A\lor C)$$
\item DeMorganove formule
$$\neg (A\lor B)\equiv \neg A\land\neg B,\quad\neg (A\land B)\equiv \neg A\lor\neg B$$
\item $$A\lor\top\equiv\top,\quad A\land\bot\equiv\bot$$
\item $$A\lor\bot\equiv A,\quad A\land\top\equiv A$$
\item komplementiranost
$$A\lor\neg A\equiv\top,\quad A\land\neg A\equiv\bot$$
\item pravilo dvostruke negacije
$$\neg\neg A\equiv A$$
\end{itemize}

Svako pravilo može se dokazati preko tablica istinitosti.

\item Iskazati i dokazati pravilo obrata po kontrapoziciji.

Vrijedi pravilo obrata po kontrapoziciji odnosno $A\Rightarrow B\equiv \neg B\Rightarrow\neg A$.

Dokažimo tvrdnju koristeći pravilo $A\Rightarrow B\equiv\neg A\lor B$. Sada iz $\neg B\Rightarrow\neg A$ imamo $$\neg B\Rightarrow\neg A\equiv\neg\neg B\lor\neg A\equiv B\lor\neg A\equiv\neg A\lor B\equiv A\Rightarrow B$$
Odnosno početna tvrdnja je istinita.


\item Definirati pojmove: tauotologija, kontradikcija.

Tautologija ... za neku formulu $P$ kažemo da je tautologija ako je uvijek istinita, u tom slučaju pišemo $\vDash P$.

Kontradikcija ... za neku formulu $Q$ kažemo da je kontradikcija ako nikad nije istinita, u tom slučaju pišemo $Q\equiv\bot$.

\item Zapisati formulama pa dokazati (algebarski i tablicom) sljedeća pravila zaključivanja: zakon isključenja trećega, pravilo silogizma, zakon neproturječnosti, zakon dvostruke negacije, pravilo kontrapozicije, zakoni apsorpcije.

U matematičkoj logici vrijede sljedeća pravila zaključivanja:
\begin{itemize}
\item zakon isključenja trećeg
$$\vDash A\lor\neg A$$
\item tranzitivnost implikacije, odnosno pravilo silogizma
$$\vDash ( A\Rightarrow B)\land(B\Rightarrow C)\Rightarrow(A\Rightarrow C)$$
\item zakon neproturječnosti
$$\vDash \neg( A\land\neg A)$$
\item zakon dvostruke negacije
$$\vDash \neg\neg A\Leftrightarrow A$$
\item pravilo kontrapozicije
$$\vDash(A\Rightarrow B)\Leftrightarrow(\neg B\Rightarrow\neg A)$$
\item zakoni apsorpcije
$$\vDash A\lor(A\land B)\Leftrightarrow A$$
$$\vDash A\land(A\lor B)\Leftrightarrow A$$
\end{itemize}

Pripadajući dokazi.

\begin{itemize}
\item zakon isključenja trećeg

\begin{displaymath}
\begin{array}{|c|c|c|}
A & \neg A & A\lor\neg A\\
\hline
\top & \bot & \top\\
\bot & \top & \top\\
\end{array}
\end{displaymath}

\item tranzitivnost implikacije, odnosno pravilo silogizma

\begin{align*}
&\quad(A\Rightarrow B)\land (B\Rightarrow C) \Rightarrow (A\Rightarrow C)\equiv \neg[(\neg A\lor B)\land (\neg B\lor C)]\lor(\neg A\lor C)\equiv\\
&\equiv\neg(\neg A\lor B)\lor\neg(\neg B\lor C)\lor(\neg A\lor C)\equiv(A\land\neg B)\lor (B\land\neg C)\lor\neg A\lor C\equiv\\
&\equiv[(A\lor\neg A)\land(\neg B\lor\neg A)]\lor [(B\lor C)\land(\neg C\lor C)]\equiv [\top\land(\neg B\lor\neg A)]\lor [(B\lor C)\land\top]\equiv\\
&\equiv \neg B\lor\neg A\lor B\lor C\equiv \neg B\lor B\lor\neg A\lor C\equiv \top\lor\neg A\lor C\equiv\top\\
\end{align*}

\item zakon neproturječnosti

Pirmjenom DeMorganovih formula vidimo da su zakon neproturječnosti i zakon isključenja trećeg isti.
$$\neg(\neg A\land A)\equiv \neg\neg A\lor \neg A\equiv A\lor\neg A$$

\item zakon dvostruke negacije

\begin{displaymath}
\begin{array}{|c|c|c|}
A & \neg\neg A & \neg\neg A\Leftrightarrow A\\
\hline
\top & \top & \top\\
\bot & \bot & \top\\
\end{array}
\end{displaymath}

ili

\begin{align*}
&\quad\neg\neg A\Leftrightarrow A\equiv A\Leftrightarrow A\equiv (A\Rightarrow A)\land(A\Rightarrow A)\equiv\\
&(\neg A\lor A)\land(\neg A\lor A)\equiv \top\land\top\equiv\top\\
\end{align*}

\item pravilo kontrapozicije

\begin{align*}
&(A\Rightarrow B)\Leftrightarrow(\neg B\Rightarrow\neg A)\equiv [\neg(A\Rightarrow B)\lor(\neg B\Rightarrow\neg A)]\land[\neg(\neg B\Rightarrow\neg A)\lor(A\Rightarrow B)]\equiv\\
&\equiv[\neg(\neg A\lor B)\lor B\lor\neg A]\land[\neg(B\lor\neg A)\lor\neg A\lor B]\equiv[(A\land\neg B)\lor B\lor\neg A]\land[(\neg B\land A)\lor\neg A\lor B]\equiv\\
&\equiv(A\land\neg B)\lor B\lor\neg A\equiv[(A\lor B)\land(\neg B\lor B)]\lor\neg A\equiv[(A\lor B)\land\top]\lor\neg A\equiv A\lor B\lor\neg A\equiv\top\\
\end{align*}

\item zakoni apsorpcije

Prvi zakon:
\begin{align*}
&\quad A\lor(A\land B)\Leftrightarrow A\equiv[A\lor(A\land B)\Rightarrow A]\land[A\Rightarrow A\lor(A\land B)]\equiv\\
&\equiv[\neg(A\lor(A\land B))\lor A]\land[\neg A\lor A\lor (A\land B)]\equiv[(\neg A\land\neg(A\land B))\lor A]\land[\top\lor(A\land B)]\equiv\\
&\equiv[(\neg A\land(\neg A\lor\neg B))\lor A]\land\top\equiv [(\neg A\land\neg A)\lor(\neg A\land\neg B)]\lor A\equiv\neg A\lor(\neg A\land\neg B)\lor A\equiv\top\\
\end{align*}

Drugi zakon:
\begin{align*}
&\quad A\land(A\lor B)\Leftrightarrow A\equiv[A\land(A\lor B)\Rightarrow A]\land[A\Rightarrow A\land(A\lor B)]\equiv\\
&\equiv[\neg(A\land(A\lor B))\lor A]\land[\neg A\lor (A\land (A\lor B))]\equiv[\neg A\lor\neg(A\lor B)\lor A]\land[(\neg A\lor A)\land(\neg A\lor A\lor B)]\equiv\\
&\equiv[\top\lor\neg(A\lor B)]\land[\top\land\top]\equiv\top\land\top\equiv\top\\
\end{align*}

\end{itemize}

\item Definirati pojam algebre sudova.

Algebra sudova je skup $S$ svih sudova zajedno s tri operacije na $S$: dvije binarne $\lor$, $\land$ i jednom unarnom $\neg$.

\item Definirati pojmove logička posljedica sudova, premise, zaključak.

Kažemo da je $A$ logička posljedica sudova $P_1, P_2, \ldots, P_n$ ako iz prepostavke da su sudovi $P_1, P_2, \ldots, P_n$ istiniti slijedi da je i sud $A$ istinit, pišemo
$$P_1, P_2, \ldots, P_n\vDash A$$

Sudovi $P_1, P_2, \ldots, P_n$ su premise, a sud $A$ je zaključak.

\item Iskazati i dokazati teorem koji karakterizira pojam logičke posljedice sudova pomoću implikacije.

Ako vrijedi $P_1, P_2, \ldots, P_n\vDash A$, onda je $\vDash P_1\land P_2\land\ldots\land P_n\Rightarrow A$ i obratno.

Dokažimo tvrdnju prvo u jednom smjeru.

Neka je $P_1, \ldots, P_n\vDash A$, tada treba dokazati $$\vDash P_1\land P_2\land\ldots\land P_n\Rightarrow A$$ 
Pretpostavimo da ono što trebamo dokazati nije tautologija, odnosno da vrijedi
$$P_1\equiv\ldots\equiv\ P_n\equiv\top,\quad A\equiv\bot$$
Ali sada uviđamo da se to protivi početnoj tvrdnji koja kaže da istinitost sudova $P_1, \ldots, P_n$ povlači istinitost suda $A$, stoga je $\vDash P_1\land P_2\land\ldots\land P_n\Rightarrow A$ tautologija.

Dokažimo sada tvrdnju u drugom smjeru.

Neka je $\vDash P_1\land P_2\land\ldots\land P_n\Rightarrow A$, tada treba dokazati $$P_1, \ldots, P_n\vDash A$$
Ako je $P_1\land P_2\land\ldots\land P_n\equiv\top$, odnosno $P_1\equiv\ldots\equiv\ P_n\equiv\top$, po pretpostavci mora vrijediti i $A\equiv\top$. Drugim riječima vrijedi $P_1, \ldots, P_n\vDash A$.

\item Iskazati i dokazati pravila modus ponens i modus tollens.

Za sudove $A$ i $B$ vrijedi $$A, A\Rightarrow B\vDash B$$ Takvo pravilo zaključivanja zove se modus ponens ili pravilo otkidanja.

Dokaz:

Ako za premise vrijedi $A\equiv\top$ i $A\Rightarrow B\equiv\top$, onda mora biti $B\equiv\top$.

Za sudove $A$ i $B$ vrijedi $$\neg B, A\Rightarrow B\vDash\neg A$$ Takvo pravilo zaključivanja zove se modus tollens.

Dokaz:

Ako po pravilu kontrapozicije zamijenimo $A\Rightarrow B$ sa $\neg B\Rightarrow\neg A$ onda tvrdnja vrijedi po pravilu modus ponens za $\neg A$ i $\neg B$.

\end{enumerate}

\color{brightlavender}
\section{Booleova algebra}
\color{white}

\begin{enumerate}
\item Definirati pojam Booleove algebre (raspisati sva svojstva).

Neka je $B$ skup u kojem su istaknuta dva različita elementa $0$ i $1$, te neka su zadane dvije binarne operacije, zbrajanje i množenje, i jedna unarna operacija $\overline{\phantom{m}}$ na $B$ . Skup $B$ zajedno s ove tri operacije zove se Booleva algebra ako su ispunjena sljedeća svojstva:

\begin{enumerate}

\item idempotentnost
$$a + a = a,\quad a\cdot a = a$$

\item asocijativnost
$$(a + b) + c = a + (b + c)$$
$$a\cdot(b\cdot c) = (a\cdot b)\cdot c$$

\item komutativnost
$$a + b = b + a,\quad a\cdot b = b\cdot a$$

\item distributivnost 
$$a\cdot(b + c) = a\cdot b + a\cdot c$$
$$a + (b\cdot c) = (a + b)\cdot (a + c)$$

\item DeMorganove formula 
$$\overline{a+b} = \overline{a}\cdot\overline{b}$$
$$\quad\overline{a\cdot b} = \overline{a}+\overline{b}$$

\item $$a + 0 = a,\quad a\cdot1 = a$$

\item $$a + 1 = 1, \quad a\cdot 0 = 0$$

\item komplementiranost 
$$a + \overline{a} = 1,\quad a\cdot\overline{a} = 0$$

\item involutivnost komplementiranja
$$\overline{\overline{a}} = a$$

\end{enumerate}

\item Dokazati da su u Booleovoj algebri nula i jedinica jedinstvene, te da vrijede pravila apsorpcije.

Pretpostavimo da $0$ i $1$ nisu jedinstveni u booleovoj algebri, odnosno da postoje $0_1, 0_2, 1_1, 1_2$. Tada iz svojstva f) vidimo:
$$0_1 + 0_2 = 0_1\quad 0_2 + 0_1 = 0_2$$
$$1_1\cdot1_2 = 1_1\quad 1_2\cdot1_1 = 1_2$$
Nadalje iz c) je očito:
$$0_1 = 0_2\quad 1_1 = 1_2$$
Drugim riječima nule i jedinice su jedinstvene u Booleovoj algebri.

\item Definirati pojam izomorfizma Booleovih algebri.

Neka su zadane dvije Booleove algebre $(B_1, +, \cdot, \overline{\phantom{m}})$ i $(B_2, +, \cdot, \overline{\phantom{m}})$. Za funkciju $f: B_1\rightarrow B_2$ kažemo da je izomorfizam Booleovih algebra $B_1$ i $B_2$ ako je bijekcija i za sve $a, b\in B_1$ vrijedi:
$$f(a\cdot b) = f(a)\cdot f(b)$$
$$f(\overline{a}) = f(a)$$

\item Dokazati da izomorfizam Booleovih algebra čuva zbrajanje, nulu i jedinicu.

Ako je $f: B_1\rightarrow B_2$ izomorfizam Booleovih algebra, onda vrijedi
$$f(a+b) = f(a) + f(b),\quad f(0_1) = 0_2, \quad f(1_1) = 1_2$$

Dokaz:

Preko e) imamo $a + b = \overline{\overline{a + b}} = \overline{\overline{a}\cdot\overline{b}}$, nadalje preko uvjeta izomorfizma dobijemo
$$f(a+b) = f(\overline{\overline{a}\cdot\overline{b}}) = \overline{f(\overline{a}\cdot\overline{b})} = \overline{f(\overline{a})\cdot f(\overline{b})} = \overline{f(\overline{a})} + \overline{f(\overline{b})} = f(\overline{\overline{a}}) + f(\overline{\overline{b}}) = f(a) + f(b)$$

Također vrijedi $$f(0_1) = f(0_1\cdot\overline{0_1}) = f(0_1)\cdot f(\overline{0_1}) = f(0_1)\cdot \overline{f(0_1)} = 0_2$$
$$f(1_1) = f(\overline{0_1}) = \overline{f(0_1) = \overline{0_2} = 1_2}$$

\end{enumerate}



\end{document}
















