 \documentclass{article}
\usepackage[total={7in, 10in}]{geometry}
\usepackage[dvipsnames]{xcolor}
\usepackage{amsmath}
\usepackage[unicode]{hyperref}
\usepackage{fancyhdr}
\usepackage{amssymb}
\usepackage{amsthm}
\usepackage[croatian]{babel}
\usepackage{multirow}


\definecolor{carminered}{rgb}{1.0, 0.0, 0.22}
\definecolor{capri}{rgb}{0.0, 0.75, 1.0}
\definecolor{brightlavender}{rgb}{0.85, 0.7, 0.95}

\title{\textbf{Diskretna matematika 120 - teorija}}
\author{Q}
\date{\today}
\begin{document}


\pagecolor{black}
\color{white}
\maketitle
\color{brightlavender}
\section{Skupovi i funkcije}
\color{white}
\pagenumbering{roman}
\fancyfoot[C]{Page \thepage\ of \pageref{LastPage}}

\begin{enumerate}

\item Objasniti što podrazumijevamo pod pojmom skupa te na koje načine možemo skup zadati.

Pod pojmom skup podrazumijevamo bilo koju množinu elemenata.

\item Navesti definicije pojmova: podskup, jednakost skupova, pravi podskup, prazan skup, partitivni skup, disjunktni skupovi.

Podskup ... skup $A$ je podskup skupa $B$ ako za bilo koji $x \in A$ također vrijedi $x \in B$ i to pišemo kao $A \subseteq B$.

Jednakost ... kažemo da su skupovi $A$ i $B$ jednaki ako vrijedi $A\subseteq B$ i $B\subseteq A$ te pišemo $A=B$.

Pravi podskup ... kažemo da je skup $A$ pravi podskup skupa $B$ ako vrijedi $A\subseteq B$ i $A\neq B$ te pišemo $A\subset B$.

Prazan skup ... onaj koji ne sadrži elemente, pišemo $\emptyset$. Za svaki skup $A$ vrijedi $\emptyset\subseteq A$.

Partitivni skup ... za bilo koji skup $A$ možemo definirati skup koji kao svoje elemente sadrži podskupove od $A$, skup svih podskupova od $A$ zovemo partitivni skup i pišemo $\mathcal{P}(A)$.

Disjunktni skupovi ... za dva skupa $A$ i $B$ vrijedi da su disjunktni ako je njihov presjek prazan skup.

\item Definirati skupovne operacije: unija, presjek, komplement, razlika, simetrična razlika, te navesti svojstva svih tih operacija.

Unija ... za dva skupa $A$ i $B$ sadržanih u univerzalnom skupu definiramo uniju skupova $A$ i $B$ kao skup svih elemenata $x$ za koje vrijedi $x\in A$ ili $x\in B$, takav skup označavamo sa $A\cup B$.

Presjek ... za dva skupa $A$ i $B$ sadržanih u univerzalnom skupu definiramo presjek skupova $A$ i $B$ kao skup svih elemenata $x$ za koje vrijedi $x\in A$ i $x\in B$, takav skup označavamo sa $A\cap B$.

Komplement ... za skup $A$ koji je sadržan u univerzalnom skupu definiramo komplement skupa $A$ kao skup svih elemenata $x$ za koje vrijedi $x\notin A$, a označavamo ${A}^\complement$ ili $\overline{A}$.

Razlika ... za dva skupa $A$ i $B$ sadržanih u univerzalnom skupu definiramo razliku skupova $A$ i $B$ kao skup svih elemenata $x$ za koje vrijedi $x\in A$ i $x\notin B$, te pišemo $A\setminus B$.

Simetrična razlika ... za dva skupa $A$ i $B$ sadržanih u univerzalnom skupu definiramo simetričnu razliku skupova $A$ i $B$ kao skup svih elemenata $x$ za koje vrijedi $x\in A$, $x\in B$ i $x\notin A\cap B$, te pišemo $A\triangle B$.

Neka su $A$, $B$ i $C\in 2^{X}$, tada vrijede sljedeća svojstva operacija nad skupovima:
\begin{itemize}
\item idempotentnost
$$A\cup A = A , \quad A\cap A = A$$
\item asocijativnost
$$A\cup(B\cup C) = (A\cup B)\cup C$$
$$A\cap(B\cap C) = (A\cap B)\cap C$$
\item komutativnost
$$A\cup B = B\cup A, \quad A\cap B = B\cap A$$
\item distributivnost
$$A\cap(B\cup C) = (A\cap B)\cup(A\cap C)$$
$$A\cup(B\cap C) = (A\cup B)\cap(A\cup C)$$
\item De Morgan
$$\overline{A\cup B} = \overline{A}\cap\overline{B}$$
$$\overline{A\cap B} = \overline{A}\cup\overline{B}$$
\item ostala pravila
$$A\cup\emptyset = A, \quad A\cap X = A$$
$$A\cup\overline{A} = X, \quad A\cap\overline{A} = \emptyset$$
$$\overline{\overline{A}} = A$$
\end{itemize}

\item Dokazati De Morganove formule za dva proizvoljna skupa.

Prema De Morganovim formulama vrijedi da za dva skupa $A$, $B \in 2^{X}$ vrijedi:
$$\overline{A\cup B} = \overline{A}\cap\overline{B}$$
$$\overline{A\cap B} = \overline{A}\cup\overline{B}$$
Dokažimo prvo prvu formulu.

Neka je $x\in\overline{A\cup B}$, odnosno $x\notin A\cup B$. Tada također vrijedi $x\notin A$ i $x\notin B$, drugim riječima $x\in \overline{A}$ i $x\in\overline{B}$ ali to samo znači $x\in\overline{A}\cap\overline{B}$ pa time vrijedi $\overline{A\cup B}\subseteq\overline{A}\cap\overline{B}$.

Sada neka je $x\in\overline{A}\cap\overline{B}$, odnosno $x\in\overline{A}$ i $x\in\overline{B}$. Uočavamo da onda vrijedi i $x\notin A$ i $x\notin B$, odnosno $x\notin A\cup B$, drugim riječima $x\in\overline{A\cup B}$ pa time vrijedi $\overline{A}\cap\overline{B}\subseteq\overline{A\cup B}$.

Pokazali smo da vrijedi
$$\overline{A\cup B}\subseteq\overline{A}\cap\overline{B}\quad i \quad \overline{A}\cap\overline{B}\subseteq\overline{A\cup B}$$
Te zaključujemo
$$\overline{A\cup B} = \overline{A}\cap\overline{B}$$
Dokažimo sada drugu formulu.

Neka je $x\in\overline{A\cap B} \Longrightarrow x\notin A\cap B \Longrightarrow x \notin A$ ili $x\notin B\Longrightarrow x\in\overline{A}$ ili $x\in\overline{B}\Longrightarrow x\in\overline{A}\cup\overline{B}$ pa vrijedi $\overline{A\cap B}\subseteq\overline{A}\cup\overline{B}$.

Sada neka je $x\in\overline{A}\cup\overline{B}\Longrightarrow x\in\overline{A}$ ili $x\in\overline{B}\Longrightarrow x\notin A$ ili $x\notin B\Longrightarrow x\notin A\cap B\Longrightarrow x\in\overline{A\cap B}$ pa vrijedi $\overline{A}\cup\overline{B}\subseteq\overline{A\cap B}$.

Pokazali smo da vrijedi
$$\overline{A\cap B}\subseteq\overline{A}\cup\overline{B} \quad i \quad \overline{A}\cup\overline{B}\subseteq\overline{A\cap B}$$
Te zaključujemo
$$\overline{A\cap B} = \overline{A}\cup\overline{B}$$

\item Definirati pojam Kartezijevog produkta  $n$ - proizvoljnih skupova.

Ako su $A_1, A_2, \ldots, A_n$ neprazni skupovi, onda definiramo Kartezijev produkt
$$A_1 \times A_2 \times\ldots\times A_n$$
kao skup svih uređenih parova $(a_1, a_2, \ldots, a_n)$ takvih da je $a_k\in A_k$ za sve $k = 1, 2, \ldots, n$

\item Definirati pojmove konačan skup, beskonačan skup, ekvipotentni skupovi, prebrojiv skup, neprebrojiv skup.

Konačan skup ... za neprazan skup $A$ kažemo da je konačan ako postoji $n\in\mathbb{N}$ i bijekcija takva da $$f: \{1, 2, \ldots, n\}\to A$$

Beskonačan skup ... za skup $A$ kažemo da je beskonačan ako nije konačan.

Ekvipotentni skupovi ... Skup $A$ je ekvipotentan sa skupom $B$ ako postoji bijekcija takva da $f: A\to B$.

Prebrojivi skup ... za beskonačan skup $A$ kažemo da je prebrojiv ako se skup njegovih elemenata može poredati u beskonačan niz $ A = \{a_1, a_2, \ldots\}$.

Neprebrojiv skup ... za beskonačan skup $A$ kažemo da je neprebrojiv ako se ne može poredati u niz.

\item Dokazati da je funkcija $f: A \to B$ injektivna akko je surjektivna, pri čemu su $A$ i $B$ konačni skupovi s jednakim brojem elemenata.

Dokažimo prvo u jednom smjeru.

Neka je $f$ injektivna funkcija, tada skupovi $A$ i $f(A)$ imaju isti broj elemenata. Također znamo da vrijedi $|A| = |B|$ pa znamo $|f(A)| = |B|$.

Za svaku funkciju vrijedi da je slika funkcije podskup kodomene odnosno $f(A)\subseteq B$, ali uz dodatan uvjet $|f(A)| = |B|$ sada imamo $f(A) = B$, odnosno funkcija je surjekcija.

Sada dokažimo u drugom smjeru.

Neka je $f$ surjektivna funkcija, tada znamo da je $f(A) = B$, odnosno $|f(A)| = |B|$. Uz uvjet $|A| = |B|$ sada imamo $|f(A)|=|A|$, drugim riječima domena i slika funkcije imaju jednak broj elemenata pa je funkcija injekcija.

\item Definirati pojam kardinalnog broja skupa te prokomentirati ekvipotentnost kod konačnih i beskonačnih skupova.

Za konačan skup $A$ s $n$ elemenata definiramo kardinalni broj skupa kao $|A| = n$.

\end{enumerate}

\color{brightlavender}
\section{Matematička logika}
\color{white}

\begin{enumerate}

\item Definirati pojmove: sud, semantička vrijednost suda.

Sud ... bilo koja rečenica koja je ili istinita ili lažna. Svakom sudu $A$ pridružujemo vrijednosti $\top$ ili $\bot$ (redom istina i lažna).

Semantička vrijednost ... vrijednost istinitosti suda, semantičku vrijednost suda $A$ označavamo sa $\tau (A)$ čija vrijednost može biti $\tau (A) = \top$ ili $\tau (A) = \bot$.

\item Definirati operacije sa sudovima: negacija, konjunkcija, disjunkcija, ekskluzivna disjunkcija, implikacija, ekvivalencija, Shefferova operacija, Lukasiewiczeva operacija.

Negacija ... negacija suda $A$ je sud koji označavamo sa $\neg A$ (čitamo: ne A), a pripadajuća tablica istinitosti je
\begin{displaymath}
\begin{array}{|c|c|}
A & \neg A\\
\hline
\top & \bot\\
\bot & \top\\
\end{array}
\end{displaymath}

Konjunkcija ... konjunkcija sudova $A$ i $B$ je sud koji označavamo sa $A\land B$ (čitamo: A i B).

Disjunkcija ... disjunkcija sudova $A$ i $B$ je sud koji označavamo sa $A\lor B$ (čitamo: A ili B).

Ekskluzivna disjunkcija ... ekskluzivna dijsunkcija sudova $A$ i $B$ je sud koji označavamo sa $A\veebar B$ (čitamo: ili A ili B).

Implikacija ... implikacija sudova $A$ i $B$ je sud koji označavamo sa $A\Rightarrow B$ (čitamo: iz A slijedi B).

Ekvivalencija ... ekvivalencija sudova $A$ i $B$ je sud koji označavamo sa $A\Leftrightarrow B$ (čitamo: A je ekvivalentan sa B).

Shefferova operacija ... shefferova operacija između sudova $A$ i $B$ je sud koji označavamo sa $A\uparrow B$ (čitamo: A šefer B), te ima značenje nije istodobno $A$ i $B$.

Lukasiewiczeva operacija ... lukasiewiczeva operacija između sudova $A$ i $B$ je sud koji označavamo sa $A\downarrow B$ (čitamo: A lukasijevič B), te ima značenje niti je $A$ niti je $B$.

Pripadajuća tablica istinitosti za operacije konjunkcije, disjunkcije, ekskluzivne disjunkcije, implikacije, ekvivalencije, Shefferove operacije i Lukasiewiczeve operacije
\begin{displaymath}
\begin{array}{|c|c|c|c|c|c|c|c|c|}
A & B & A\land B & A\lor B & A\veebar B & A\Rightarrow B & A\Leftrightarrow B & A\uparrow B & A\downarrow B\\
\hline
\bot & \bot & \bot & \bot & \bot & \top & \top & \top & \top\\
\bot & \top & \bot & \top & \top & \top & \bot & \top & \bot\\
\top & \bot & \bot & \top & \top & \bot & \bot & \top & \bot\\
\top & \top & \top & \top & \bot & \top & \top & \bot & \bot\\
\end{array}
\end{displaymath}


\item Definirati pojam logičke ekvivalentnosti dviju logičkih formula te navesti poredak logičkih operacija po snazi vezivanja.

Logička ekvivalentnosti ... Kažemo da su dvije formule $P$ i $Q$ algebre sudova logički ekvivalentne ako imaju isti broj varijabla i podjednake tablice istinitosti, te pišemo $P\equiv Q$.

Poredak logičkih operacija po opadajućoj snazi vezivanja je
$$\neg,\quad\land,\quad\lor,\quad\Rightarrow,\quad\Leftrightarrow$$

\item Definirati pojmove generatora algebre sudova i baze algebre sudova.

Generator algebre sudova ... sistem izvodnica (generatora) algebre sudova je skup Booleovih operacija algebre sudova pomoću kojih se može napisati bilo koja formula algebre sudova.

Baza algebre sudova ... Minimalni sistem izvodnica algebre sudova zovemo bazom algebre sudova.

\item Iskazati i dokazati teorem koji daje svojstva operacija sa sudovima.

Vrijede sljedeća pravila algebre sudova
\begin{itemize}
\item idempotentnost operacija disjunkcije i konjunkcije
$$A\lor A\equiv A,\quad A\land A\equiv A$$
\item asocijativnost
$$(A\lor B)\lor C\equiv A\lor (B\lor C)$$
$$(A\land B)\land C\equiv A\land (B\land C)$$
\item komutativnost
$$A\lor B\equiv B\lor A,\quad A\land B\equiv B\land A$$
\item distributivnost
$$A\land (B\lor C)\equiv (A\land B)\lor (A\land C)$$
$$A\lor (B\land C)\equiv (A\lor B)\land (A\lor C)$$
\item DeMorganove formule
$$\neg (A\lor B)\equiv \neg A\land\neg B,\quad\neg (A\land B)\equiv \neg A\lor\neg B$$
\item $$A\lor\top\equiv\top,\quad A\land\bot\equiv\bot$$
\item $$A\lor\bot\equiv A,\quad A\land\top\equiv A$$
\item komplementiranost
$$A\lor\neg A\equiv\top,\quad A\land\neg A\equiv\bot$$
\item pravilo dvostruke negacije
$$\neg\neg A\equiv A$$
\end{itemize}

FALI DOKAZZZZZZZZZZZZZZZZZZZZZZZZZZZZZZZZZZZZZZZZZZZZZZZZZZZZZZZZZZZZZZZZZZZZZ

\item Iskazati i dokazati pravilo obrata po kontrapoziciji.

\end{enumerate}



\end{document}
















