 \documentclass{article}
\usepackage[total={7in, 10in}]{geometry}
\usepackage[dvipsnames]{xcolor}
\usepackage{amsmath}
\usepackage[unicode]{hyperref}
\usepackage{fancyhdr}
\usepackage{amssymb}
\usepackage{amsthm}
\usepackage[croatian]{babel}
\usepackage{multirow}


\definecolor{carminered}{rgb}{1.0, 0.0, 0.22}
\definecolor{capri}{rgb}{0.0, 0.75, 1.0}
\definecolor{brightlavender}{rgb}{0.85, 0.7, 0.95}

\title{\textbf{Prvi Tjedan}}
\author{Q}
\date{}
\begin{document}


\pagecolor{black}
\color{white}
\maketitle

\begin{enumerate}

\item Objasniti što podrazumijevamo pod pojmom skupa te na koje načine možemo skup zadati.

Pod pojmom skup podrazumijevamo bilo koju množinu elemenata.

\item Navesti definicije pojmova: podskup, jednakost skupova, pravi podskup, prazan skup, partitivni skup, disjunktni skupovi.

Podskup ... skup $A$ je podskup skupa $B$ ako za bilo koji $x \in A$ također vrijedi $x \in B$ i to pišemo kao $A \subseteq B$.

Jednakost ... kažemo da su skupovi $A$ i $B$ jednaki ako vrijedi $A\subseteq B$ i $B\subseteq A$ te pišemo $A=B$.

Pravi podskup ... kažemo da je skup $A$ pravi podskup skupa $B$ ako vrijedi $A\subseteq B$ i $A\neq B$ te pišemo $A\subset B$.

Prazan skup ... onaj koji ne sadrži elemente, pišemo $\emptyset$. Za svaki skup $A$ vrijedi $\emptyset\subseteq A$.

Partitivni skup ... za bilo koji skup $A$ možemo definirati skup koji kao svoje elemente sadrži podskupove od $A$, skup svih podskupova od $A$ zovemo partitivni skup i pišemo $\mathcal{P}(A)$.

Disjunktni skupovi ... za dva skupa $A$ i $B$ vrijedi da su disjunktni ako je njihov presjek prazan skup.

\item Definirati skupovne operacije: unija, presjek, komplement, razlika, simetrična razlika, te navesti svojstva svih tih operacija.

Unija ... za dva skupa $A$ i $B$ sadržanih u univerzalnom skupu definiramo uniju skupova $A$ i $B$ kao skup svih elemenata $x$ za koje vrijedi $x\in A$ ili $x\in B$, takav skup označavamo sa $A\cup B$.

Presjek ... za dva skupa $A$ i $B$ sadržanih u univerzalnom skupu definiramo presjek skupova $A$ i $B$ kao skup svih elemenata $x$ za koje vrijedi $x\in A$ i $x\in B$, takav skup označavamo sa $A\cap B$.

Komplement ... za skup $A$ koji je sadržan u univerzalnom skupu definiramo komplement skupa $A$ kao skup svih elemenata $x$ za koje vrijedi $x\notin A$, a označavamo ${A}^\complement$ ili $\overline{A}$.

Razlika ... za dva skupa $A$ i $B$ sadržanih u univerzalnom skupu definiramo razliku skupova $A$ i $B$ kao skup svih elemenata $x$ za koje vrijedi $x\in A$ i $x\notin B$, te pišemo $A\setminus B$.

Simetrična razlika ... za dva skupa $A$ i $B$ sadržanih u univerzalnom skupu definiramo simetričnu razliku skupova $A$ i $B$ kao skup svih elemenata $x$ za koje vrijedi $x\in A$, $x\in B$ i $x\notin A\cap B$, te pišemo $A\triangle B$.

Neka su $A$, $B$ i $C\in 2^{X}$, tada vrijede sljedeća svojstva operacija nad skupovima:
\begin{itemize}
\item idempotentnost
$$A\cup A = A , \quad A\cap A = A$$
\item asocijativnost
$$A\cup(B\cup C) = (A\cup B)\cup C$$
$$A\cap(B\cap C) = (A\cap B)\cap C$$
\item komutativnost
$$A\cup B = B\cup A, \quad A\cap B = B\cap A$$
\item distributivnost
$$A\cap(B\cup C) = (A\cap B)\cup(A\cap C)$$
$$A\cup(B\cap C) = (A\cup B)\cap(A\cup C)$$
\item De Morgan
$$\overline{A\cup B} = \overline{A}\cap\overline{B}$$
$$\overline{A\cap B} = \overline{A}\cup\overline{B}$$
\item ostala pravila
$$A\cup\emptyset = A, \quad A\cap X = A$$
$$A\cup\overline{A} = X, \quad A\cap\overline{A} = \emptyset$$
$$\overline{\overline{A}} = A$$
\end{itemize}

\item Dokazati De Morganove formule za dva proizvoljna skupa.

Prema De Morganovim formulama vrijedi da za dva skupa $A$, $B \in 2^{X}$ vrijedi:
$$\overline{A\cup B} = \overline{A}\cap\overline{B}$$
$$\overline{A\cap B} = \overline{A}\cup\overline{B}$$
Dokažimo prvo prvu formulu.

Neka je $x\in\overline{A\cup B}$, odnosno $x\notin A\cup B$. Tada također vrijedi $x\notin A$ i $x\notin B$, drugim riječima $x\in \overline{A}$ i $x\in\overline{B}$ ali to samo znači $x\in\overline{A}\cap\overline{B}$ pa time vrijedi $\overline{A\cup B}\subseteq\overline{A}\cap\overline{B}$.

Sada neka je $x\in\overline{A}\cap\overline{B}$, odnosno $x\in\overline{A}$ i $x\in\overline{B}$. Uočavamo da onda vrijedi i $x\notin A$ i $x\notin B$, odnosno $x\notin A\cup B$, drugim riječima $x\in\overline{A\cup B}$ pa time vrijedi $\overline{A}\cap\overline{B}\subseteq\overline{A\cup B}$.

Pokazali smo da vrijedi
$$\overline{A\cup B}\subseteq\overline{A}\cap\overline{B}\quad i \quad \overline{A}\cap\overline{B}\subseteq\overline{A\cup B}$$
Te zaključujemo
$$\overline{A\cup B} = \overline{A}\cap\overline{B}$$
Dokažimo sada drugu formulu.

Neka je $x\in\overline{A\cap B} \Longrightarrow x\notin A\cap B \Longrightarrow x \notin A$ ili $x\notin B\Longrightarrow x\in\overline{A}$ ili $x\in\overline{B}\Longrightarrow x\in\overline{A}\cup\overline{B}$ pa vrijedi $\overline{A\cap B}\subseteq\overline{A}\cup\overline{B}$.

Sada neka je $x\in\overline{A}\cup\overline{B}\Longrightarrow x\in\overline{A}$ ili $x\in\overline{B}\Longrightarrow x\notin A$ ili $x\notin B\Longrightarrow x\notin A\cap B\Longrightarrow x\in\overline{A\cap B}$ pa vrijedi $\overline{A}\cup\overline{B}\subseteq\overline{A\cap B}$.

Pokazali smo da vrijedi
$$\overline{A\cap B}\subseteq\overline{A}\cup\overline{B} \quad i \quad \overline{A}\cup\overline{B}\subseteq\overline{A\cap B}$$
Te zaključujemo
$$\overline{A\cap B} = \overline{A}\cup\overline{B}$$

\item Definirati pojam Kartezijevog produkta  $n$ - proizvoljnih skupova.

Ako su $A_1, A_2, \ldots, A_n$ neprazni skupovi, onda definiramo Kartezijev produkt
$$A_1 \times A_2 \times\ldots\times A_n$$
kao skup svih uređenih parova $(a_1, a_2, \ldots, a_n)$ takvih da je $a_k\in A_k$ za sve $k = 1, 2, \ldots, n$

\item Definirati pojmove konačan skup, beskonačan skup, ekvipotentni skupovi, prebrojiv skup, neprebrojiv skup.

Konačan skup ... za neprazan skup $A$ kažemo da je konačan ako postoji $n\in\mathbb{N}$ i bijekcija takva da $$f: \{1, 2, \ldots, n\}\to A$$

Beskonačan skup ... za skup $A$ kažemo da je beskonačan ako nije konačan.

Ekvipotentni skupovi ... Skup $A$ je ekvipotentan sa skupom $B$ ako postoji bijekcija takva da $f: A\to B$.

Prebrojivi skup ... za beskonačan skup $A$ kažemo da je prebrojiv ako se skup njegovih elemenata može poredati u beskonačan niz $ A = \{a_1, a_2, \ldots\}$.

Neprebrojiv skup ... za beskonačan skup $A$ kažemo da je neprebrojiv ako se ne može poredati u niz.

\item Dokazati da je funkcija $f: A \to B$ injektivna akko je surjektivna, pri čemu su $A$ i $B$ konačni skupovi s jednakim brojem elemenata.

Dokažimo prvo u jednom smjeru.

Neka je $f$ injektivna funkcija, tada skupovi $A$ i $f(A)$ imaju isti broj elemenata. Također znamo da vrijedi $|A| = |B|$ pa znamo $|f(A)| = |B|$.

Za svaku funkciju vrijedi da je slika funkcije podskup kodomene odnosno $f(A)\subseteq B$, ali uz dodatan uvjet $|f(A)| = |B|$ sada imamo $f(A) = B$, odnosno funkcija je surjekcija.

Sada dokažimo u drugom smjeru.

Neka je $f$ surjektivna funkcija, tada znamo da je $f(A) = B$, odnosno $|f(A)| = |B|$. Uz uvjet $|A| = |B|$ sada imamo $|f(A)|=|A|$, drugim riječima domena i slika funkcije imaju jednak broj elemenata pa je funkcija injekcija.

\item Definirati pojam kardinalnog broja skupa te prokomentirati ekvipotentnost kod konačnih i beskonačnih skupova.

Za konačan skup $A$ s $n$ elemenata definiramo kardinalni broj skupa kao $|A| = n$.

\end{enumerate}

\end{document}
