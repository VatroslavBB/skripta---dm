 \documentclass{article}
\usepackage[total={7in, 10in}]{geometry}
\usepackage[dvipsnames]{xcolor}
\usepackage{amsmath}
\usepackage[unicode]{hyperref}
\usepackage{fancyhdr}
\usepackage{amssymb}
\usepackage{amsthm}
\usepackage[croatian]{babel}
\usepackage{multirow}


\definecolor{carminered}{rgb}{1.0, 0.0, 0.22}
\definecolor{capri}{rgb}{0.0, 0.75, 1.0}
\definecolor{brightlavender}{rgb}{0.85, 0.7, 0.95}

\title{\textbf{Drugi tjedan}}
\author{Q}
\date{}
\begin{document}


\pagecolor{black}
\color{white}
\maketitle

\begin{enumerate}

\item Definirati pojmove: sud, semantička vrijednost suda.

Sud ... bilo koja rečenica koja je ili istinita ili lažna. Svakom sudu $A$ pridružujemo vrijednosti $\top$ ili $\bot$ (redom istina i lažna).

Semantička vrijednost ... vrijednost istinitosti suda, semantičku vrijednost suda $A$ označavamo sa $\tau (A)$ čija vrijednost može biti $\tau (A) = \top$ ili $\tau (A) = \bot$.

\item Definirati operacije sa sudovima: negacija, konjunkcija, disjunkcija, ekskluzivna disjunkcija, implikacija, ekvivalencija, Shefferova operacija, Lukasiewiczeva operacija.

Negacija ... negacija suda $A$ je sud koji označavamo sa $\neg A$ (čitamo: ne A), a pripadajuća tablica istinitosti je
\begin{displaymath}
\begin{array}{|c|c|}
A & \neg A\\
\hline
\top & \bot\\
\bot & \top\\
\end{array}
\end{displaymath}

Konjunkcija ... konjunkcija sudova $A$ i $B$ je sud koji označavamo sa $A\land B$ (čitamo: A i B).

Disjunkcija ... disjunkcija sudova $A$ i $B$ je sud koji označavamo sa $A\lor B$ (čitamo: A ili B).

Ekskluzivna disjunkcija ... ekskluzivna dijsunkcija sudova $A$ i $B$ je sud koji označavamo sa $A\veebar B$ (čitamo: ili A ili B).

Implikacija ... implikacija sudova $A$ i $B$ je sud koji označavamo sa $A\Rightarrow B$ (čitamo: iz A slijedi B).

Ekvivalencija ... ekvivalencija sudova $A$ i $B$ je sud koji označavamo sa $A\Leftrightarrow B$ (čitamo: A je ekvivalentan sa B).

Shefferova operacija ... shefferova operacija između sudova $A$ i $B$ je sud koji označavamo sa $A\uparrow B$ (čitamo: A šefer B), te ima značenje nije istodobno $A$ i $B$.

Lukasiewiczeva operacija ... lukasiewiczeva operacija između sudova $A$ i $B$ je sud koji označavamo sa $A\downarrow B$ (čitamo: A lukasijevič B), te ima značenje niti je $A$ niti je $B$.

Pripadajuća tablica istinitosti za operacije konjunkcije, disjunkcije, ekskluzivne disjunkcije, implikacije, ekvivalencije, Shefferove operacije i Lukasiewiczeve operacije
\begin{displaymath}
\begin{array}{|c|c|c|c|c|c|c|c|c|}
A & B & A\land B & A\lor B & A\veebar B & A\Rightarrow B & A\Leftrightarrow B & A\uparrow B & A\downarrow B\\
\hline
\bot & \bot & \bot & \bot & \bot & \top & \top & \top & \top\\
\bot & \top & \bot & \top & \top & \top & \bot & \top & \bot\\
\top & \bot & \bot & \top & \top & \bot & \bot & \top & \bot\\
\top & \top & \top & \top & \bot & \top & \top & \bot & \bot\\
\end{array}
\end{displaymath}


\item Definirati pojam logičke ekvivalentnosti dviju logičkih formula te navesti poredak logičkih operacija po snazi vezivanja.

Logička ekvivalentnosti ... Kažemo da su dvije formule $P$ i $Q$ algebre sudova logički ekvivalentne ako imaju isti broj varijabla i podjednake tablice istinitosti, te pišemo $P\equiv Q$.

Poredak logičkih operacija po opadajućoj snazi vezivanja je
$$\neg,\quad\land,\quad\lor,\quad\Rightarrow,\quad\Leftrightarrow$$

\item Definirati pojmove generatora algebre sudova i baze algebre sudova.

Generator algebre sudova ... sistem izvodnica (generatora) algebre sudova je skup Booleovih operacija algebre sudova pomoću kojih se može napisati bilo koja formula algebre sudova.

Baza algebre sudova ... Minimalni sistem izvodnica algebre sudova zovemo bazom algebre sudova.

\item Iskazati i dokazati teorem koji daje svojstva operacija sa sudovima.

Vrijede sljedeća pravila algebre sudova
\begin{itemize}
\item idempotentnost operacija disjunkcije i konjunkcije
$$A\lor A\equiv A,\quad A\land A\equiv A$$
\item asocijativnost
$$(A\lor B)\lor C\equiv A\lor (B\lor C)$$
$$(A\land B)\land C\equiv A\land (B\land C)$$
\item komutativnost
$$A\lor B\equiv B\lor A,\quad A\land B\equiv B\land A$$
\item distributivnost
$$A\land (B\lor C)\equiv (A\land B)\lor (A\land C)$$
$$A\lor (B\land C)\equiv (A\lor B)\land (A\lor C)$$
\item DeMorganove formule
$$\neg (A\lor B)\equiv \neg A\land\neg B,\quad\neg (A\land B)\equiv \neg A\lor\neg B$$
\item $$A\lor\top\equiv\top,\quad A\land\bot\equiv\bot$$
\item $$A\lor\bot\equiv A,\quad A\land\top\equiv A$$
\item komplementiranost
$$A\lor\neg A\equiv\top,\quad A\land\neg A\equiv\bot$$
\item pravilo dvostruke negacije
$$\neg\neg A\equiv A$$
\end{itemize}

FALI DOKAZZZZZZZZZZZZZZZZZZZZZZZZZZZZZZZZZZZZZZZZZZZZZZZZZZZZZZZZZZZZZZZZZZZZZ

\item Iskazati i dokazati pravilo obrata po kontrapoziciji.

\end{enumerate}

\end{document}

