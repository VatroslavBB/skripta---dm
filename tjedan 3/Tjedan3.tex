 \documentclass{article}
\usepackage[total={7in, 10in}]{geometry}
\usepackage[dvipsnames]{xcolor}
\usepackage{amsmath}
\usepackage[unicode]{hyperref}
\usepackage{fancyhdr}
\usepackage{amssymb}
\usepackage{amsthm}
\usepackage[croatian]{babel}
\usepackage{multirow}


\definecolor{carminered}{rgb}{1.0, 0.0, 0.22}
\definecolor{capri}{rgb}{0.0, 0.75, 1.0}
\definecolor{brightlavender}{rgb}{0.85, 0.7, 0.95}

\title{\textbf{Treći tjedan}}
\author{Q}
\date{}
\begin{document}
\pagecolor{black}
\color{white}
\maketitle

\begin{enumerate}

\item Definirati pojmove: tauotologija, kontradikcija.

Tautologija ... za neku formulu $P$ kažemo da je tautologija ako je uvijek istinita, u tom slučaju pišemo $\vDash P$.

Kontradikcija ... za neku formulu $Q$ kažemo da je kontradikcija ako nikad nije istinita, u tom slučaju pišemo $Q\equiv\bot$.

\item Zapisati formulama pa dokazati (algebarski i tablicom) sljedeća pravila zaključivanja: zakon isključenja trećega, pravilo silogizma, zakon neproturječnosti, zakon dvostruke negacije, pravilo kontrapozicije, zakoni apsorpcije.

U matematičkoj logici vrijede sljedeća pravila zaključivanja:
\begin{itemize}
\item zakon isključenja trećeg
$$\vDash A\lor\neg A$$
\item tranzitivnost implikacije, odnosno pravilo silogizma
$$\vDash ( A\Rightarrow B)\land(B\Rightarrow C)\Rightarrow(A\Rightarrow C)$$
\item zakon proturječnosti
$$\vDash \neg( A\land\neg A)$$
\item zakon dvostruke negacije
$$\vDash \neg\neg A\Leftrightarrow A$$
\item pravilo kontrapozicije
$$\vDash(A\Rightarrow B)\Leftrightarrow(\neg B\Rightarrow A)$$
\item zakoni apsorpcije
$$\vDash A\lor(A\land B)\Leftrightarrow A$$
$$\vDash A\land(A\lor B)\Leftrightarrow A$$
\end{itemize}

\color{carminered}
triba jos dokazat
\color{white}

\item Definirati pojam algebre sudova.

Algebra sudova je skup $S$ svih sudova zajedno s tri operacije na $S$: dvije binarne $\lor$, $\land$ i jednom unarnom $\neg$.

\item Definirati pojmove logička posljedica sudova, premise, zaključak.

Kažemo da je $A$ logička posljedica sudova $P_1, P_2, \ldots, P_n$ ako iz prepostavke da su sudovi $P_1, P_2, \ldots, P_n$ istiniti slijedi da je i sud $A$ istinit, pišemo
$$P_1, P_2, \ldots, P_n\vDash A$$

Sudovi $P_1, P_2, \ldots, P_n$ su premise, a sud $A$ je zaključak.

\item Iskazati i dokazati teorem koji karakterizira pojam logičke posljedice sudova pomoću implikacije.

Ako vrijedi $P_1, P_2, \ldots, P_n\vDash A$, onda je $\vDash P_1\land P_2\land\ldots\land P_n\Rightarrow A$ i obratno.

Dokažimo tvrdnju prvo u jednom smjeru.

Neka je $P_1, \ldots, P_n\vdash A$, tada treba dokazati $$\vDash P_1\land P_2\land\ldots\land P_n\Rightarrow A$$ 
Iz pretpostavke vidimo da istinitost sudova $P_1, \ldots, P_n$ povlači istinitost suda $A$, pa uzmimo da vrijedi 
$$P_1\equiv P_2\equiv\ldots\equiv P_n\equiv\top$$ 
Ako pretpostavimo da je tvrdnja lažna onda mora vrijediti
$$\vDash P_1\land P_2\land\ldots\land P_n\quad A\equiv\bot$$
odnosno,
$$P_1\equiv P_2\equiv\ldots\equiv P_n\equiv\top\quad A\equiv\bot$$
što je u kontradikciji s pretpostavkom, stoga tvrdnja mora biti istinita.

Dokažimo sada tvrdnju u drugom smjeru.

\item Iskazati i dokazati pravila modus ponens i modus tollens.

\item Definirati pojam Booleove algebre (raspisati sva svojstva).

\item Dokazati da su u Booleovoj algebri nula i jedinica jedinstvene, te da vrijede pravila apsorpcije.

\item Definirati pojam izomorfizma Booleovih algebri.

\item Dokazati da izomorfizam Booleovih algebra čuva zbrajanje, nulu i jedinicu.

\end{enumerate}

\end{document}

